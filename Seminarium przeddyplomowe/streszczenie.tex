\documentclass[11pt]{article}
\usepackage[a4paper,top=2cm,bottom=2cm,left=3cm,right=3cm]{geometry}
\usepackage{setspace}
\usepackage{graphicx}
\usepackage{indentfirst}
\usepackage[T1]{fontenc}
\usepackage[polish]{babel}

\title{\normalsize Praca dyplomowa magisterska\\[4ex] \LARGE Analiza i porównanie wydajności gry mobilnej i internetowej wytworzonej z wykorzystaniem frameworka Flutter i silnika gry Flame}
\author{inż. Andrzej Kapczyński\\[3ex]
  Promotor: dr inż. Marcin Borowski}

\begin{document}
\maketitle

\section*{Streszczenie} 
W obliczu dynamicznego rozwoju technologii mobilnych i aplikacji internetowych, istnieje kluczowa potrzeba zrozumienia oraz oceny efektywności stosowanych narzędzi programistycznych. Na współczesnym rynku istnieje szereg technologii umożliwiających tworzenie aplikacji multiplatformowych, co znacznie przyspiesza proces realizacji projektów. Warto podkreślić, że mimo możliwości tworzenia aplikacji za pomocą jednego kodu źródłowego dla różnych platform, technologie te mogą różnić się pod względem wydajności na poszczególnych środowiskach. Różnice te wynikają często z indywidualnego procesu kompilacji i optymalizacji dostosowanej do konkretnych środowisk docelowych. Praca podjęta została w celu analizy i porównania wydajności gry wytworzonej przy użyciu najbardziej popularnego frameworka do tworzenia aplikacji wieloplatformowych w 2023 roku. Wybór tej konkretnej technologii był kierowany faktem, że pokrywa ona większość rynku takich aplikacji, co sprawia, że stanowi ona istotny punkt odniesienia.\\

Celem pracy w części badawczej jest przeprowadzenie dogłębnej analizy oraz porównania wydajności gry mobilnej i internetowej, zrealizowanych przy użyciu frameworka Flutter oraz silnika gry Flame. W ramach części projektowej zaplanowano zaprojektowanie i implementację gry platformowej 2D, stanowiącej obiekt testowy dla przeprowadzanych badań.\\

Planowany sposób weryfikacji podejścia obejmuje przeprowadzenie eksperymentów, które dostarczą solidnych podstaw do wyciągnięcia wniosków szczegółowych dotyczących wydajności gry na obu platformach. Eksperymenty te będą obejmować testy wydajnościowe, analizujące parametry takie jak liczba klatek na sekundę, zużycie pamięci RAM oraz obciążenie procesora.\\

Weryfikacja podejścia zostanie przeprowadzona etapowo, uwzględniając różne scenariusze działania gry oraz zmienne warunki środowiskowe. Wartości uzyskane w wyniku eksperymentów pozwolą na dokładną analizę wpływu wybranych technologii na wydajność gry mobilnej i internetowej.\\

Wymienione wcześniej parametry, znacznie wpływają na efektywność aplikacji. Wraz z wynikami testów, praca będzie w stanie w pełni zrealizować cel postawiony przed projektem, dostarczając wartościowych informacji dla praktyków i badaczy zajmujących się tworzeniem aplikacji wieloplatformowych.\\

\end{document}